\documentclass{ctexart}
\usepackage{graphicx, hyperref, amsmath, amsfonts}

\begin{document}

这是对前面内容的一个比较大的整合——mukism。

在一切实在层次,复杂系统都只是完全的由实践定义。从拟合的角度对于整体系统\textbf{U}的抽象才能成功。

一定程度上,对系统的抽象类似抽象代数的手法,譬如$\mathbf{im} f$与$\mathbf{U} / \mathbf{ker} f$。可以从层次的角度理解,前者将考虑的$\mathbf{S}$的上层隔离,后者将$\mathbf{S}$的下层的效应模糊化。

一个理解上的问题是,易于从还原论的理解而认为总是底层的效应影响上层的结果。但是这个理论在层次关系上其实根本不存在因果论,就如同\textit{Maxwell's Formula}并不意味着电场和磁场存在因果关系一样。

进而考虑$\mathbf{S}$的内部结构,实际上是两个概念的相互关系,即$\mathbf{M}$与$\mathbf{K}$。这两者的关系本身具有强对称性,但是$\mathbf{M}$与下层系统关联,$\mathbf{K}$与上层系统关联,两者的全部不对称性来源于此。

\end{document}
