\documentclass{ctexart}
\usepackage{graphicx, hyperref}

\begin{document}

\textit{小六年级下期末测试模拟一作文}

家乡是无甚平原的,我至今只去过一次北方——北京。印象中总是南方的山峦,翠映山水。

一次回家乡登山扫墓,记得不是清明,但家乡在重山之间,云雾笼罩,更爬升了四百余米,时间是春日,风却有秋日的清清愁愁。雨不小的下了起来,却看不见雨滴,只见脸上蒙了一层薄雾,惆惆的,稠稠的。灰灰的水田上圈圈的波纹。我写过篇深圳梧桐烟云的诗,心里朦朦胧胧不知是否借了某位先生的文字。其中有一句“烟雨茫茫空乱”,自觉应是在此处更合适了。

云灰蒙蒙,雨灰蒙蒙,这世界灰蒙蒙的,裹上一层灰蒙蒙的罩,更加的灰蒙蒙了。风清新新的,雨清新新的,水也清新新的,愁愁的潺潺的流着,忧忧郁郁的,清清愁愁的。柔柔的,让人爱着,好像躺在柔柔的草地上,郁郁青青的,裹着我,绕着我,让我溺死在着无穷无尽的雨中。

我爱的一个景色,上世纪一十年代上海的小街是个印象罢了,或许记错了时间地点,或许根本没有这样的地儿),两边的楼灰蒙蒙的,地上灰蒙蒙的,下着灰蒙蒙的淅淅沥沥的小雨,人们撑着灰蒙蒙的油纸伞往来,为了各自的事而奔忙不知是谁在灰蒙蒙的老相机后记录下这一刻,凝然在一张老照片中。在青青郁郁的山中,有灰蒙蒙的雨,没有灰蒙蒙的油纸伞,灰蒙蒙的世界笼罩住我,抚摸着我……

这天不是清明,更不是秋,却有着灰蒙蒙的秋雨与秋风,有着最爱的愁愁郁郁的景。

\end{document}
