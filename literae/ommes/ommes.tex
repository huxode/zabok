\documentclass{ctexart}
\usepackage{graphicx, hyperref, epigraph, etoolbox}

\renewcommand{\epigraphflush}{center}

\begin{document}

\epigraph{
    泽雅的秋天也与春天没什么变化,
    不过是秋天罢了。
}{}

泽雅的秋天也与春天没什么变化,山间仍是葱葱的云雾缭绕,杉杉的行道树倒是已泛黄,点缀在青青的林中;水田的稻谷大都已收,是恰恰才能遇见余下的灿烂的稻田。

豆腐鲞是依然的熏人的香味。软色的片煎出噼里啪啦的滴滴油泡,饼上是泛出闪耀的金黄,嚼一口,是郁郁的气息与片片落入口中,凝聚的是山水的香味。

水库旁盘绕的九曲勾勒出一个亚特兰蒂斯式沉没的模样,也刻画出历史与回忆的曲线。今日只剩下了昔日的记忆可追寻的踪迹,是在环环的翠竹中的波光粼粼的明珠。

到龙井与齐云山的感觉仿佛是相似的。温润之州是无白石峰三千米草原的凛冽,山谷中挟着秋雨的风亦能霜结是人的脸庞。隐约破碎的雾气映着洁白的连廊与亭仿佛遥不可及,霜下闪耀的花朵为龙井再抹上一层鲜艳的色彩。

记起一片灰蒙蒙清翠翠的风景,如云彩在天中翻滚。溪下的河谷是寄托了乡土生活滴滴的脚步,如同梯田中穿过的每一滴水,挂下回念的步伐。

是生在楼厦间的孩子呀,却更加在山水间留下的回忆。泽雅这波澜的凹凸,又承载了多少游子的乡愁?当你告诉孩子这花草的故事,是铭记了你心中何等的回忆?当孩子听见是这片扎根的土地,又是一厢如何的心灵?

于是那个秋,我坐在黄透了的水杉中间,前方是公路边的崖壁;我是被清翠翠的山水与云雾包围着,是踏在我祖祖辈辈的土壤上啊。一条陡峭的溪谷尽头小小的村庄,有破败了的老屋与焕新的黄氏的祠堂,山后已被野草几近吞噬的小道的凉亭上捐者的石板上,刻着祖父的名字。我嚼着口中汁黄脆香的豆腐鲞,望着我深深扎根的地方——

泽雅的秋天也与春天没什么,

只不过是泽雅罢了。


\end{document}
