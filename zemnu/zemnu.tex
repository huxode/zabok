\documentclass{ctexart}
\usepackage{graphicx, hyperref, amssymb}

\begin{document}

\section{色空}

如瓶盛水:

眼见瓶身,从而见中空之处,从而方得盛水。

所以,色依空,空依色;空即是色,色即是空;非空非色,即色即空

\section{道名}

道是轮转规律;名是信息,是时间百态。

道通于空,名通于色。道名空色,相生相济相涵。

\section{基数}

易于证明,信息或文字$\mathbb{D}$的集合的基数为$\aleph_0$。存在猜测思维能够部分突破这个极限,使其能容许的状态的基数达到$\mathfrak{c}$。

\section{贯通}

从二元论的角度能够构建一张谱表。

\begin{table}[h]
    \centering
    \begin{tabular}{|l|l|}
        \hline
        空 & 色 \\
        \hline
        道 & 名 \\
        \hline
        $\aleph_0$ & $\mathfrak{c}$ \\
        \hline
    \end{tabular}
\end{table}

这意味着这三个方面具有极高的相通性。

\end{document}
