\documentclass{ctexart}
\usepackage{graphicx, hyperref}

\begin{document}

文化的体现是为传承。

文化体现的根本是文化思维结构。文化的形成就是文化思维结构的形成。文化自然随时间发展,思维结构亦如此。

文化的形成过程中的核心因素是物理条件。物理条件确定生产条件。生产条件决定思维结构的下层路径。下层路径易于传承,普遍,必要,是思维结构的基础。下层路径依赖于物理条件,离开了物理条件就容易变化与消失。

文化逐渐的不仅在生产过程也在一般生活中形成。形成阶段的终点是形成文化哲学。文化哲学决定思维结构的上层路径。上层路径决定高阶内容的传承与构建,从而决定一切的文化成果。

文化一旦发生融合(侵略,交流等)与割裂(地域,独立等),上层路径与下层路径就会发生一定的反应。若发生融合,既然融合两方的物理条件有差异,所以下层路径不变仍然独立发展,上层路径融合;若发生割裂,既然割裂双方具有不同的物理条件,所以下层路径割裂发展产生大的不同,上层路径只有微小的不同或仍然相同。

文化内部或有独立传承的小族群。亦形成类似文化的形态,可能是文化的胚胎。

语言是文化思维结构的直接反映。语言与文化相辅相成共同发展。语言随着文化的形成而形成,也随文化的融合割裂而融合割裂。所以语言在文化的分析中至关重要,能够跟踪语言的发展历程,就能跟踪文化的发展历程。

\end{document}
