\documentclass{ctexart}
\usepackage{graphicx, hyperref}

\begin{document}

\section{1.0}

历史并不完全取决于经济与政治因素,但是人类历史进程宏观的方方面面都与经济政治共进,其根本是人类的思维的宏观现象。

Thomas Kuhn的《科学革命的结构》(\textit{The Structure of Scientific Revolutions})的理论由此可以推广。

西方社会在19世纪末20世纪初进入了一个完备化的思潮中。在这一时期许多“最终解决方案”被提出,典型的如Russell与Whitehead的《数学原理》(\textit{Principia Mathematica}),19世纪末对“物理学大厦已经建成”的说法。这种思维暂且称之为完备主义。

完备主义实际上是帝国主义的雏形。这并不是说完备主义存在某种罪恶,而是其客观上与帝国主义的思维存在大的相似性。西方社会的经济,政治与学术齐头并进,在世纪之交达到高峰,即完备主义,其在政治上体现为帝国主义。

完备主义在诞生时就注定其命运,因为“完备”本身就具有极大的不稳定性与不确定性。物理学的完备主义在20世纪初相对论与量子论的研究崩塌,数学的完备主义则同样在20世纪初数理逻辑的研究中崩塌,政治的完备主义的崩塌过程自然引起了两次世界大战以及之后的冷战,并随着社会的发展而愈来愈淡化。

这样的崩塌过程自然的引发新一轮的革命。学术的大厦遭到了重构,政治的格局亦经历了重塑。这一定程度上塑造了现世的极端人道主义与所谓的“联合国”主义。

完备本身是追求确定的,但是其又必然是不确定的。完备主义的革命在新时代中产生了怀疑的矛盾的,不可知的迷茫的情结。这一情结贯穿了20世纪后期一系列的怀疑与疯狂,而且更加的在21世纪信息结构的推波助澜下增广。

\section{2023.4.3 update}

这一时代事件的主旋律在于“确定性”,其本质是人类思维的确定性革命。

这一时代或者被称为“20世纪的偏离”(约1880-1940)。过高的确定性与极端性偏离了人类思维的稳定性。正如瀑布落下时形成复杂激荡的涡流,其导致“20世纪的混乱”(约1940-今)。与水流的类比可以引申为复杂系统的势的思路。

\end{document}
