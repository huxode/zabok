\documentclass{ctexart}
\usepackage{graphicx, hyperref}

\begin{document}

复杂系统具有连续结构。

无论使用地缘,或者其他任何的组织方式,都可以构建一个系统相关的场。场的元素不可能简单的作为一个标量,而应该构成一个特殊的内积空间。通常讨论的“东方”“西方”可以理解为空间中的一组无关组,但是显然该空间远不止二维,所以其不构成一组基。

场随时间变化。对于生命系统存在渲染效应,是进化论逻辑的延伸。有理由认为渲染对于其他的复杂系统同样有效,这当是场作用中重要的一环。对于地球上构建的复杂系统,通常渲染是作用在地缘关系上的。很有可能心智的渲染是作用在类比联想关系上的。

渲染可比喻为一个状态的流动,自然携带一个速度场或者环境场。环境场依赖其联系层复杂系统变化。这种联系形成一种复杂系统互相影响的网络,这是通过对环境场的干预产生的。

一般的复杂系统与量子力学理论具有一定的相似相通性。其向量结构与渲染效应都具有合适的类比。有理由认为渲染的初级量化与Schrödinger方程具有相似性。

\end{document}
