\documentclass{ctexart}
\usepackage{graphicx, hyperref, booktabs}

\begin{document}

\section{\textbf{GOUPU}}

人类通过某种方式团结在一起或获得共情,赋予这样的方式一个名字:共径。共径的组合构成一个层次结构。

低层次的共径的分类较为模糊,例如宗族,土地,共同利益等或许是一个层次或相近层次的共径,而由于不同的文化结构而分化。

高层次的共径大致类似一个递进关系
\begin{tabular}{cc}
    \toprule   
        层次 & 共径
    \midrule 
        $h \leq H$ & 低层次共径 \\
        $h = H + 1$ & 语言 \\
        $h = H + 2$ & 宗教 \\
        $h = H + 3$ & 阶级 \\
    \bottomrule
\end{tabular}

\textbf{problem: 层次概念的极端模糊性:程度?关系度?}

\section{\textbf{PLOBI}}

变概率空间:实验行为改变$Pr$的概率空间
混沌?

\textbf{problem: 内容思考不足}

\section{\textbf{EDUDE}}

政府的需求:经济发展.国防
学术界的需求:解决学术问题
社会的需求:社会稳定性
其他参与者的需求(纯粹利己方面):自身利益

对于教育:
忽略了人自身的需求

\textbf{problem: 内容思考不足}

\end{document}

